\section{Viscoelastic self sustaining process paper
plan}\label{viscoelastic-self-sustaining-process-paper-plan}

\subsection{Introduction}\label{introduction}

\begin{itemize}
\itemsep1pt\parskip0pt\parsep0pt
\item
  Why have I done this?
\item
  Which other work inspires this?

  \begin{itemize}
  \itemsep1pt\parskip0pt\parsep0pt
  \item
    Waleffe
  \item
    People talking about lift up
  \item
    people talking about What sustains turbulence
  \end{itemize}
\item
  Summary of the point

  \begin{itemize}
  \itemsep1pt\parskip0pt\parsep0pt
  \item
    get streaks in purely elastic regime
  \item
    streaks are unstable
  \item
    Wavy, but also not wavy instabilities
  \item
    Free slip boundary conditions turn out to be important
  \end{itemize}
\end{itemize}

\subsection{Methods}\label{methods}

\begin{itemize}
\itemsep1pt\parskip0pt\parsep0pt
\item
  Fourier - Chebyshev - Fourier decomposition
\item
  Newton Rhaphson method on the solution forced with rolls
\item
  Solutions give a streaky streamwise velocity profile
\item
  Then do linear stability analysis on these solutions with a
  disturbance of a specific kx fourier mode.
\item
  Repeat this for all kx to build up a picture of the dispersion
  relation
\item
  Examine eigenmodes that correspond to these eigenvalues
\end{itemize}

\subsection{Results}\label{results}

\subsubsection{Streaky profile}\label{streaky-profile}

\begin{itemize}
\itemsep1pt\parskip0pt\parsep0pt
\item
  See the lift up mechanism when the amplitude of the rolls is
  sufficiently large.
\item
  Just like Waleffe's result at low Wi
\item
  Rather then large inflection in velocity you see regions of high
  Normal stress difference.

  \begin{itemize}
  \itemsep1pt\parskip0pt\parsep0pt
  \item
    Corresponds to shearing of polymers?
  \end{itemize}
\end{itemize}

\subsubsection{dispersion relations}\label{dispersion-relations}

\begin{itemize}
\itemsep1pt\parskip0pt\parsep0pt
\item
  As Reynold's number decreases, the instability noticed by Waleffe goes
  away.
\item
  At intemediate Reynold's numbers and Weissenberg numbers (elasticity
  \textasciitilde{} 0.1 ?) the instability disappears.
\item
  At low Re and high Wi (elasticity \textasciitilde{} 1000?), we see a
  purely elastic instability arise at very low kx.
\item
  This instability is increased by reducing Reynold's number or
  increasing the Wiessenberg number.
\item
  The instability plateaus at a Wiessenberg number of \textasciitilde{}
  18 we think.
\end{itemize}

\subsubsection{eigenmodes}\label{eigenmodes}

\begin{itemize}
\itemsep1pt\parskip0pt\parsep0pt
\item
  Although the instability is large at the walls, most of the gradients
  in v0,v1,w1 take place away from the walls.
\item
  It is the gradients that are responsible for the variations in the
  normal stress and so can reinforce the rolls, as in the original self
  sustaining process of Waleffe.
\end{itemize}

\subsection{No slip case}\label{no-slip-case}

\begin{itemize}
\itemsep1pt\parskip0pt\parsep0pt
\item
  Without slip at the boundaries the instability appears infinitely
  amplified
\item
  {[}Show var slip graph{]}
\item
  A possible explanation for this is that the instability moves towards
  the walls and can no longer be resolved due to large gradients in the
  velocity.
\end{itemize}

\subsection{Discussion}\label{discussion}

\begin{itemize}
\itemsep1pt\parskip0pt\parsep0pt
\item
  A purely elastic instability has been uncovered in a situation
  analogous to that of the self sustaining process, a fundemental
  component coherent structure in Newtonian turbulence in wall bounded
  shear flow.
\item
  The instability is very different to that of the Newtonian case, it
  remains to be shown that it can reinforce the streamwise rolls and
  complete the self sustaining process.
\item
  It occurs at zero kx, meaning it does not correspond to a
  kelvin-helmholtz like instability of the streaks in the streamwise
  direction?
\item
  The potential importance of free slip boundary conditions, known for
  ages for polymeric fluids, has been explored. Free slip and no slip
  seem to behave differently.
\end{itemize}

\subsection{Conclusions}\label{conclusions}

\begin{itemize}
\itemsep1pt\parskip0pt\parsep0pt
\item
  I am so glad this paper is over.
\end{itemize}
