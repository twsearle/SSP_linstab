%	paper.tex	Toby Searle
%	Last modified: Wed 14 May 18:27:29 2014
%	using jfm template

\NeedsTeXFormat{LaTeX2e}

\documentclass{jfm}

\usepackage{graphicx}
\usepackage{natbib}
%%%%%%%%%%%Added to template:
\usepackage{caption, subcaption, mathtools} % add my own packages
\graphicspath{ {./images/} }
\newcommand\Wi{\mbox{\textit{Wi}}}
\newcommand{\dt}[1]{\frac{d #1}{d t}} %time deriv
\newcommand{\dy}[1]{\frac{\partial #1}{\partial y}}
\newcommand{\me}{\mathrm{e}}
\newcommand{\xmean}[1]{\overline{#1}^{x}}
%%%%%%%%%%%%%%%

% See if the author has AMS Euler fonts installed: If they have, attempt
% to use the 'upmath' package to provide upright math.
\ifCUPmtlplainloaded \else
  \checkfont{eurm10}
  \iffontfound
    \IfFileExists{upmath.sty}
      {\typeout{^^JFound AMS Euler Roman fonts on the system,
                   using the 'upmath' package.^^J}%
       \usepackage{upmath}}
      {\typeout{^^JFound AMS Euler Roman fonts on the system, but you
                   dont seem to have the}%
       \typeout{'upmath' package installed. JFM.cls can take advantage
                 of these fonts,^^Jif you use 'upmath' package.^^J}%
       \providecommand\upi{\pi}%
      }
  \else
    \providecommand\upi{\pi}%
  \fi
\fi

% See if the author has AMS symbol fonts installed: If they have, attempt
% to use the 'amssymb' package to provide the AMS symbol characters.

\ifCUPmtlplainloaded \else
  \checkfont{msam10}
  \iffontfound
    \IfFileExists{amssymb.sty}
      {\typeout{^^JFound AMS Symbol fonts on the system, using the
                'amssymb' package.^^J}%
       \usepackage{amssymb}%
       \let\le=\leqslant  \let\leq=\leqslant
       \let\ge=\geqslant  \let\geq=\geqslant
      }{}
  \fi
\fi

% See if the author has the AMS 'amsbsy' package installed: If they have,
% use it to provide better bold math support (with \boldsymbol).

\ifCUPmtlplainloaded \else
  \IfFileExists{amsbsy.sty}
    {\typeout{^^JFound the 'amsbsy' package on the system, using it.^^J}%
     \usepackage{amsbsy}}
    {\providecommand\boldsymbol[1]{\mbox{\boldmath $##1$}}}
\fi

%%% Example macros (some are not used in this sample file) %%%

% For units of measure
\newcommand\dynpercm{\nobreak\mbox{$\;$dyn\,cm$^{-1}$}}
\newcommand\cmpermin{\nobreak\mbox{$\;$cm\,min$^{-1}$}}

% Various bold symbols
\providecommand\bnabla{\boldsymbol{\nabla}}
\providecommand\bcdot{\boldsymbol{\cdot}}
\newcommand\biS{\boldsymbol{S}}
\newcommand\etb{\boldsymbol{\eta}}

% For multiletter symbols
\newcommand\Real{\mbox{Re}} % cf plain TeX's \Re and Reynolds number
\newcommand\Imag{\mbox{Im}} % cf plain TeX's \Im
\newcommand\Rey{\mbox{\textit{Re}}}  % Reynolds number
\newcommand\Pran{\mbox{\textit{Pr}}} % Prandtl number, cf TeX's \Pr product
\newcommand\Pen{\mbox{\textit{Pe}}}  % Peclet number
\newcommand\Ai{\mbox{Ai}}            % Airy function
\newcommand\Bi{\mbox{Bi}}            % Airy function

% For sans serif characters:
% The following macros are setup in JFM.cls for sans-serif fonts in text
% and math.  If you use these macros in your article, the required fonts
% will be substitued when you article is typeset by the typesetter.
%
% \textsfi, \mathsfi   : sans-serif slanted
% \textsfb, \mathsfb   : sans-serif bold
% \textsfbi, \mathsfbi : sans-serif bold slanted (doesnt exist in CM fonts)
%
% For san-serif roman use \textsf and \mathsf as normal.
%
\newcommand\ssC{\mathsf{C}}    % for sans serif C
\newcommand\sfsP{\mathsfi{P}}  % for sans serif sloping P
\newcommand\slsQ{\mathsfbi{Q}} % for sans serif bold-sloping Q

% Hat position
\newcommand\hatp{\skew3\hat{p}}      % p with hat
\newcommand\hatR{\skew3\hat{R}}      % R with hat
\newcommand\hatRR{\skew3\hat{\hatR}} % R with 2 hats
\newcommand\doubletildesigma{\skew2\tilde{\skew2\tilde{\Sigma}}}
%       italic Sigma with double tilde

% array strut to make delimiters come out right size both ends
\newsavebox{\astrutbox}
\sbox{\astrutbox}{\rule[-5pt]{0pt}{20pt}}
\newcommand{\astrut}{\usebox{\astrutbox}}

\newcommand\GaPQ{\ensuremath{G_a(P,Q)}}
\newcommand\GsPQ{\ensuremath{G_s(P,Q)}}
\newcommand\p{\ensuremath{\partial}}
\newcommand\tti{\ensuremath{\rightarrow\infty}}
\newcommand\kgd{\ensuremath{k\gamma d}}
\newcommand\shalf{\ensuremath{{\scriptstyle\frac{1}{2}}}}
\newcommand\sh{\ensuremath{^{\shalf}}}
\newcommand\smh{\ensuremath{^{-\shalf}}}
\newcommand\squart{\ensuremath{{\textstyle\frac{1}{4}}}}
\newcommand\thalf{\ensuremath{{\textstyle\frac{1}{2}}}}
\newcommand\Gat{\ensuremath{\widetilde{G_a}}}
\newcommand\ttz{\ensuremath{\rightarrow 0}}
\newcommand\ndq{\ensuremath{\frac{\mbox{$\partial$}}{\mbox{$\partial$} n_q}}}
\newcommand\sumjm{\ensuremath{\sum_{j=1}^{M}}}
\newcommand\pvi{\ensuremath{\int_0^{\infty}%
  \mskip \ifCUPmtlplainloaded -30mu\else -33mu\fi -\quad}}

\newcommand\etal{\mbox{\textit{et al.}}}
\newcommand\etc{etc.\ }
\newcommand\eg{e.g.\ }


\newtheorem{lemma}{Lemma}
\newtheorem{corollary}{Corollary}

\title[A purely elastic self sustaining process in plane couette flow]{A purely elastic self sustaining process in plane couette flow}

\author[T. W. Searle and A. N. Morozov]%
{T. W. Searle$^1$ and A. N. Morozov$^1$%
  \thanks{Email address for correspondence: T.W.Searle@sms.ed.ac.uk},\ns
}

% NOTE: A full address must be provided: department, university/institution, town/city, zipcode/postcode, country.
\affiliation{$^1$SUPA, School of Physics and Astronomy, University of Edinburgh, Mayfield Road,
Edinburgh, EH9 3JZ, UK\\[\affilskip]
}

\pubyear{2013}
\volume{650}
\pagerange{119--126}
% Do not enter received and revised dates. These will be entered by the editorial office.
\date{?; revised ?; accepted ?. - To be entered by editorial office}
%\setcounter{page}{1}
\begin{document}

\maketitle

\begin{abstract}
  Abstract goes here. Abstract goes here. Viscoelastic Kelvin-Helmholtz instability. 
\end{abstract}

\begin{keywords}
Authors should not enter keywords on the manuscript, as these must be chosen by
the author during the online submission process and will then be added during
the typesetting process (see
http://journals.cambridge.org/data/\linebreak[3]relatedlink/jfm-\linebreak[3]keywords.pdf
for the full list)
\end{keywords}

\section{Introduction}\label{sec:intro}

We present a study of a purely elastic analogue of the Newtonian
self-sustaining process presented in \cite{Waleffe1997}.

In 1997 Waleffe published a seminal article that demonstrated the existence of
an exact coherent state that sustains turbulence in Newtonian plane Couette
fluid flow. This solution consists of a self-sustaining process (SSP) of three
phases. First the lift-up mechanism (sometimes called shear tilting), first
studied by \cite{Ellingsen1975, Landahl1980}.  induced by streamwise rolls
moves fluid between the plates. Due to conservation of momentum, this creates
spanwise streaks in the streamwise velocity. In the second phase, these streaks
become wavy in the streamwise direction. This is due to a linear instability in
the streaky profile from a Kelvin-Helmholtz like effect as the streaks shear
with the rest of the fluid. Finally, the nonlinear self-interaction terms from
the wavy disturbance regenerate the original streamwise rolls.

This mechanism has proven very important. Since then, similar structures have
been found in Newtonian pipe flow (both in numerical simulations in
\cite{Wedin2004} and in experiments in \cite{Hof2004}). The success of these
exact solutions has fuelled the move towards an understanding of Newtonian
turbulence in terms of these exact solutions, with the turbulent attactor
constructed out of many exact solutions which become more densely packed as the
Reynold's number increases. More recently, recurrent solutions have been
discovered which give statistics which closely match those of experimental
turbulent flows (Initially in \cite{Kawahara2001}, with others discovered in
pipe flows in \cite{Kreilos2012} and two dimensional Kolmogorov flow by
\cite{Chandler2013}). According to the emerging picture, a turbulent fluid
trajectory in phase space `pinballs' between these states, spending most of its
time very close to one of these solutions ( see e.g. \cite{Cvitanovic2013}).

In 1990 Larson, Shaqfeh and Muller provided theoretical and experimental
evidence of purely elastic instability \cite{Larson1990}. Later,
\cite{Byars1994} and \cite{Groisman2000} found spiral patterns and
then full turbulence in a low Reynold's number plate-plate flow. The mechanism
behind this instability is similar to the Weissenberg effect first observed by
Karl \cite{Weissenberg1947}. These instabilities are part of a body of evidence
which supports the existence of a different kind of turbulence, purely elastic
turbulence, where the instabilities are generated and sustained by this Normal
stress mechanism. Curved streamlines in the fluid lead to large normal
stresses, which create more curved streamlines. It is this feedback which it is
supposed will lead to a state of continuous instability.  We think that this
mechanism might provide a viscoelastic analogue of the Newtonian SSP.

Studies have already been carried out on a viscoelastic self-sustaining process
( \cite{Stone2004, Stone2002, Sureshkumar1997}). However, these studies were
concerned with high Reynold's number viscoelastic flows. Primarly they were
interested in discovering how a small concentration of polymeric fluid can
bring about a reduction in viscous drag in a fluid. They sought to explain this
viscoelastic drag reduction by an appeal to the effect of the polymers on
Waleffe's Newtonian self-sustaining process. These studies found that there is
a minimum Reynold's number below which the SSP solution ceases to exist. We are
find results which are consistent with this, however, the streak instabilty
reappears at much lower Reynold's number, in the purely elastic regime. Neither
the simulations in \cite{Sureshkumar1997} nor in \cite{Stone2004} give any
results for $Re\sim1$. They also introduce an unphysical approximation to the
equations via a stress diffusion term. It is possible that in the purely
elastic limit this stress diffusion will remove the large stress gradients
necessary for instability. 

The above studies find that the addition of a small amount of viscoelastic
fluid serves to reduce the strength of the Newtonian exact coherent structures.
The mechanism responsible for the effect of the polymeric fluid on the SSP is
that of vortex unwinding. The polymeric stress opposes the nonlinearities that
produce the vortices in the third phase of the process. In our study, the
polymeric stresses can provide nonlinear terms for the production of vortices
at very low Reynold's number, but they do so via the nonlinearities in the
Oldroyd-B equation rather than the Navier-Stokes equation.

We expect chaotic systems to be structured by exact solutions to the equations.
Given the above results, it seems that these exact solutions cannot be found by
continuation from their counterparts for Newtonian flows. Instead, we have
attempted to construct a viscoelastic analogue of the SSP as a means to
constructing purely elastic exact coherent structures.  The presence of these
exact coherent structures is strongly suggested by earlier experimental
(\cite{Pan2013, Samanta2013}) and numerical results (\cite{Morozov2007}?).

We use the Oldroyd-B model for the constituitive equation for the polymeric
component of the fluid and the full Navier-Stokes equations,
\begin{align}
    \Rey \left[ \dt{\mathbf{v}} + \mathbf{v} \cdot \nabla  \mathbf{v} \right] &= - \nabla p + \beta \nabla^{2} \mathbf{v} + (1-\beta)\nabla \cdot \mathbf{\tau} \label{eq:Navier-Stokes}\\
    \nabla \cdot \mathbf{v} &= 0 \label{eq:incompressibility}\\
    \tau + Wi\left[\frac{\partial \tau}{\partial t} + \mathbf{v} \cdot \nabla \tau - (\nabla v)^T \tau - \tau (\nabla v) \right] &= \left(\nabla \mathbf{v}\right)^{T} + \nabla{\mathbf{v}}\label{eq:Oldroyd-B} \\
\end{align}
where $\Rey$ is the Reynold's number, $\Wi$ is the Weissenberg number, $\nabla$
is the gradient operator, $\nabla^{2}$ is the Laplacian and $
\overset{\nabla}\tau$ is the upper convected derivative of the stress tensor.
By direct numerical simulation, we investigate the three phases of the process.
First we use a Newton-Raphson method to show that streaky flow is a solution at
low $\Rey$ to the these equations when forced by streamwise independent rolls.
Then we do a linear stability analysis of the streaky flow to show that these
streaks are unstable to perturbations both dependent and independent of the
streamwise wavenumber $k_x$. Finally we investigate how the eigenvectors which
correspond to these instabilities will affect the base flow, completing the
SSP.

At this point it is important to note that we are using one of the simplest
models of a viscoelastic fluid, the Oldroyd-B fluid. They include all of the
ingredients for the normal stress effect believed to be responsible for
viscoelastic turbulence.  However, it does not capture the shear thinning
behaviour. It is also well known that under extensional flows (when $\lambda
\dot{\epsilon} > 1/2$) the Oldroyd-B model breaks down. In the self-sustaining
process as we have outlined it, there ought to be no extensional flows, so this
danger turns out to be irrelevent (is this actually true?). Support for this
point of view can be found in Stone \cite{Stone2004} where they find that in
their simulations the polymer extension never exceeds 10\% of the contour
length. 

The viscoelastic streamwise vortex forcing system we use here does produce
streaks and these streaks do become unstable. The instabilities we observe can
be wavy in the streamwise direction, but tend to have very long wavelengths.
The instability that we observe is enough to produce feedback on the original
rolls of the correct sign and symmetry. The instability can only be tracked for
free slip boundary conditions, suggesting that the presence or otherwise of
polymer slip at the walls might be important in this system. 

\section{Streaky profile}\label{sec:streaks}

For the first step of the self-sustaining process we need to form a streaky
base profile in the velocity. One way of forming these streaks is to start with large
fluctuations in the wall-normal stress, $\tau_{yy}$. This stress is then
advected by the mean shear to give fluctuations in the stress in the streamwise
direction and therefore a streaky base profile. An alternative is to use
streamwise independent rolls and their associated pressure gradient. Similar to
Waleffe \cite{Waleffe1997}, this gives,

\begin{equation}
    \mathit{Re} \left[ V(y,z) \frac{\partial U}{\partial y} + W(y,z) \frac{\partial U}{\partial z} \right] = \beta \nabla^{2}_{2} U + \left(1-\beta \right) \left(\frac{\partial T_{xy}}{\partial y} + \frac{\partial T_{xz}}{\partial z} \right) \label{eq:streaks}
\end{equation}

We use a Chebyshev-Fourier decomposition, with Chebyshev polynomials in the
wall-normal (y) direction and Fourier modes in the spanwise (z) direction. We
then solve for the streamwise velocity and Stresses of the base profile using
equation \ref{eq:streaks} and the Oldroyd-B equation via a Newton-Raphson
method. In order to decompose the system onto the computational grid, we take a
Fourier and Chebyshev transform of the variables,

\begin{equation}
    \check{G}(y,z) = \sum_{n=-N}^{N} \sum_{m=0}^{M-1} G_{m,n} \me^{i n \gamma z} T_{m}(y) \label{eq:decomp}
\end{equation}
where $g$ stands for any of the base profile variables
($U,V,W,\boldsymbol{\tau}_{ij}$) in the problem and $T_{m}(y)$ is the
\textit{m}th Chebyshev polynomial of the first kind.

The system is driven by the standard no-slip boundary conditions on the
streamwise velocity at the walls,
\begin{equation}
    U(\pm 1) = 1
\end{equation}
and forcing terms introduced via fixing of the wall-normal and spanwise base
profile velocities,
\begin{align}
    V(y,z) &= V_0 \hat{v}(y) cos(\gamma z) \\
    W(y,z) &= -\frac{V_0}{\gamma} \dy{\hat{v}} \sin(\gamma z) 
\end{align}
Where,
\begin{equation}
    \hat{v}(y) = \frac{\cos(py)}{\cos(p)} - \frac{\cosh(\gamma y)}{cosh(\gamma)} 
\end {equation}
$p$ is given by solutions to $p\tan p + \gamma \tanh \gamma = 0$ and controls
the number of rolls in the wall-normal direction. These velocities are a guess
for the rolls derived from the lowest order eigenmode of the operator
$\nabla^{4}$, precisely the same rolls used in \cite{Waleffe1997}. This provides
us with the streaky profile shown below.

After using the above Newton-Rhaphson method we obtain a streaky profile in the
fluid. At high $\Rey$ we find streaks in the streamwise velocity similar to
those in \cite{Waleffe1997}. However, as we decrease the Reynold's number for
constant and large Weissenberg number, we find that the streaks in the
streamwise velocity become less pronounced. This can be explained by
considering that at low Reynold's number the fluid has a lower inertia, so it
is more difficult for the lift up effect to produce streaks in the streamwise
velocity. 

Although we do not see streaks in the streamwise velocity, we do see them in
the first normal stress difference, $N_{1} = T_{xx} - T_{yy}$ (figure
\ref{fig:pf_N1_map}). These streaks appear in much the same place as the
streamwise velocity streaks appear in the Newtonian self-sustaining process. As
noted earlier, instabilities in viscoelastic fluids are brought about by large
changes in the first normal stress difference, since this brings about polymer
stretching of the kind seen in the Weissenberg effect. It is important to note
that the purely elastic 1st normal stress difference shows very large gradients
near the wall, suggesting that resolving the instability in this base profile
will be more difficult than in the Newtonian case.

%\begin{figure}
%    \showthe\columnwidth %use this to determine size of figures.
%    \includegraphics[width=\textwidth]{./figures/vel_map_Rey002_Wi2_amp002}
%    \label{fig:pf_vel_map}
%\end{figure}

\begin{figure}
    \centering
    \begin{subfigure}[b]{0.48\textwidth}
	\includegraphics[width=\textwidth]{./figures/vorticity_map_Re200_Wi2_amp002}
	\label{fig:pf_vort_map}
    \end{subfigure}
    ~
    \begin{subfigure}[b]{0.48\textwidth}
	\includegraphics[width=\textwidth]{./figures/N1_map_Rey002_Wi2_amp002}
	\label{fig:pf_N1_map}
    \end{subfigure}
    \caption{
	a) The magnitude of the vorticity of the fluid at $\Rey = 200$, $\Wi =
	2.0$, amp=0.02. b) The first normal stress difference in the polymeric
	fluid at $\Rey = 0.02$, $\Wi = 2$, amp=0.02. The Newtonian vorticity
	and purely elastic first normal stress difference have large gradients
	in similar regions. However, the first normal stress difference also
	shows large gradients next to the wall. 
    } 
\end{figure}

\section{Linear Stability Analysis}\label{sec:linear_stability}

Having obtained the full base profile of the problem, we then performed a
linear stability analysis, to look for instabilities that might produce
waviness in the streaks. These wavy instabilities are responsible for
sustaining the exact coherent state in the Newtonian version of the process.
This gives us the linear stability equations,
\begin{align}
(\nabla \mathbf{v})^T + (\nabla \mathbf{v}) =& \tau + Wi \left[ \frac{\partial \tau}{\partial t} + (\mathbf{V} \cdot \nabla) \tau + (\mathbf{v} \cdot \nabla) T \right.\\ 
& \left. \phantom{\tau + Wi} - (\nabla \mathbf{v})^T \tau - (\nabla \mathbf{v})^T T - \tau (\nabla \mathbf{V}) - T (\nabla \mathbf{v}) \right]
\end{align}

We decompose the disturbance velocities in the same basis as above (equation
\ref{eq:decomp}), but include streamwise dependence
\begin{equation}
    \check{g}(x,y,z,t) = \sum_{n=-N}^{N} \sum_{m=0}^{M+1} g_{m,n}  T_{m}(y) \me^{i(k_{x} x + n \gamma z)} \me{i\lambda t}\label{eq:decomp_disturbances}
\end{equation}
where $g$ can be any of the disturbance variables
($u,v,w,p,\boldsymbol{\tau}_{ij}$). To increase the numerical stability of the
problem, we use free slip boundary conditions on the disturbance velocities,
$\dy{u}(\pm1) = v(\pm 1) = \dy{w}(\pm1) = 0$.

The linearised system of equations now gives an eigenvalue problem for the
growth rate of the instability $\lambda$ at every streamwise wavenumber of the
disturbance.

The solution to the eigenvalue problem provides spectra for which eigenmodes
with positive growth rates are unstable.

As the Reynold's number is decreased, we find that the base profile becomes
more stable. The dispersion relation is reduced in height and moves to lower
streamwise wavenumbers. By about $\Rey = 100$ the base profile has become
completely stable. The Newtonian instability is no longer present at this
Reynold's number. However, once the Reynolds number becomes negligible in
comparison to the Weissenberg number, we begin to see a purely elastic
instability arise at very low streamwise wavenumber
\ref{fig:dispersions_varyRe}. This purely elastic instability is hugely
amplified by further reductions in the Reynold's number.

\begin{figure}
%    \showthe\columnwidth %use this to determine size of figures.
    \centering
    \includegraphics[width=\textwidth]{./figures/dispersions_varyRe}
    \caption{
	Dispersion relations as the Reynold's number is decreased at $\Wi = 2$,
	$\beta=0.1$ and amp = 0.02. A clear instability can be seen at low
	$k_{x}$ in the purely elastic regime.
    }
    \label{fig:dispersions_varyRe}
\end{figure}

We can further examine how the purely elastic instability changes with changing
Weissenberg number. We find that it grows as the Weissenberg number increases,
and saturates by around $\Wi ~ 20$. Doubling the amplitude of the rolls
increases the width of the instablity by about a third and the height, 3 fold.  

\begin{figure}
%    \showthe\columnwidth %use this to determine size of figures.
    \centering
    \includegraphics[width=\textwidth]{./figures/dispersions_varyWi}
    \caption{
	Dispersion relations as the $\Wi$ is decreased at $\Rey = 0.001$,
	$\beta=0.1$ and amp = 0.02. The instability grows and then saturates.
    }
    \label{fig:dispersions_varyWi}
\end{figure}

\subsection{Cauchy boundary conditions}\label{sec:Cauchy}

To investigate the affect of free slip boundary conditions on linear stability
analysis, we implemented Cauchy boundary conditions,
\begin{align}
    \alpha\left.\dy{u}\right|_{y=\pm1} + (1-\alpha)u(y=\pm1) &= 0 \\
    v(y=\pm1) &=0 \\
    \alpha\left.\dy{w}\right|_{y=\pm1} + (1-\alpha)w(y=\pm1) &= 0 
\end{align}
Where $\alpha$ is a control parameter which controls the relative strength of
the free slip to no-slip conditions. (Slip length?)

We found that on decreasing $\alpha$ the instability grows in strength and
broadens (figure \ref{fig:dispersions_varslip}). If we examine the dependence
of the instability at one wavenumber, $k_{x}=0.01$, it appears that the growth
rate of the no-slip system tends to infinity as $\alpha \rightarrow 0$ (figure
\ref{fig:evaldiverg_varslip}).

\begin{figure}
%    \showthe\columnwidth %use this to determine size of figures.
    \centering
    \includegraphics[width=\textwidth]{./figures/dispersions_varslip}
    \caption{
	Dispersion relations as $\alpha$ the slip parameter, is reduced. As the
	system becomes closer to no-slip at the walls the instability grows.
    }
    \label{fig:dispersions_varslip}
\end{figure}

\section{Nonlinear feedback on the rolls}\label{sec:nonlinear_feedback}

In Fabian Waleffe's treatment of the Newtonian version of this self-sustaining
process \cite{Waleffe1997}, he found that the self-interaction terms due to the
eigenvector of the instability give feedback on the original rolls. Although
not conclusive, this is good evidence that the cycle might be closed. The
eigenvector of the viscoelastic instability is quite different to that of the
Newtonian instability. Although the components of the instability are large at
the walls, the velocity is less important for the instability of the purely
elastic fluid than the first normal stress difference. We find that $N_{1}$ is
large away from the walls, another encouraging sign for a viscoelastic
self-sustaining process (\ref{fig:eigenmode_visco}). 

\begin{figure}
%    \showthe\columnwidth %use this to determine size of figures.
    \centering
    \includegraphics[width=\textwidth]{./figures/eigenmode_visco}
    \caption{
	Eigenmodes of the first normal stress difference of the viscoelastic
	instability at $\Wi = 2.0$, $\Rey = 0.02$, $\beta=0.1$, amp=0.02 and
	$k_x = 0.01$.
    }
    \label{fig:eigenmode_visco}
\end{figure}

A schematic analysis for the nonlinear feedback similar to that performed by
Waleffe \cite{Waleffe1997} requires solving for the rolls due to forcing from the
nonlinear terms in the Oldroyd-B equation as well as the Newtonian nonlinear terms,

\begin{equation}
    \nabla^{4}_{2} \xmean{\Psi}  = \mathit{Re} \xmean{(\mathbf{v} \cdot \nabla)
    \nabla^{2} \Psi} - \left(1-\beta \right) \left(\xmean{\frac{\partial^{2}
    T_{yy}}{\partial y \partial z}} - \xmean{\frac{\partial^{2}
    T_{zz}}{\partial y \partial z}} + \xmean{\frac{\partial^{2}
    T_{yz}}{\partial z \partial z}} - \xmean{\frac{\partial^{2}
    T_{yz}}{\partial y \partial y}} \right)
\end{equation}

\begin{align}
\xmean{T_{yz}} =& Wi \left[ -\xmean{(\mathbf{v} \cdot \nabla) \tau_{yz}} -
\xmean{\frac{\partial u}{\partial x}\tau_{yz}} + \xmean{\frac{\partial
v}{\partial x} \tau_{xy}} \right. \\
& \phantom{Wi } \left. + \xmean{\frac{\partial v}{\partial z} \tau_{zz}} +
\xmean{\frac{\partial w}{\partial x} \tau_{xy}} + \xmean{\frac{\partial
w}{\partial y} \tau_{yy}} \right] \\
\xmean{T_{yy}} =& -Wi(\xmean{\mathbf{v}\cdot \nabla) \tau_{yy}} + 2Wi\left[
\xmean{\frac{\partial v}{\partial x}\tau_{xy}} + \xmean{\frac{\partial
v}{\partial y} \tau_{yy}} + \xmean{\frac{\partial v}{\partial z} \tau_{yz}}
\right] \\
\xmean{T_{zz}} =& -Wi\xmean{(\mathbf{v}\cdot \nabla) \tau_{zz}} + 2Wi\left[
\xmean{\frac{\partial w}{\partial x}\tau_{xz}} + \xmean{\frac{\partial
w}{\partial y} \tau_{yz}} + \xmean{\frac{\partial w}{\partial z} \tau_{zz}}
\right] \\
V = \frac{\partial \Psi}{\partial z}
\end{align}

\begin{figure}
%    \showthe\columnwidth %use this to determine size of figures.
    \centering
    \includegraphics[width=\textwidth]{./figures/nonlin_forcing}
    \caption{
	In blue the original amplitude of the rolls in the wall normal
	direction. Solid line is the real part, dashed is the imaginary part.
	In red the rolls obtained via nonlinear feedback from the viscoelastic
	instability at $\Wi = 2.0$, $\Rey = 0.01$, $\beta=0.1$, amp=0.02 and
	$k_x = 0.001$. Dot-dashed is the real part and dashed is the imaginary
	part.
    }
    \label{fig:eigenmode_visco}
\end{figure}

Presumably, the no slip instability will have large gradients in the velocities
and stresses at the walls as well as those in the bulk. Nevertheless, the shape
of the rolls is largly independent of the forcing, and so we don't expect any
change to the nonlinear feedback.

When we completely remove slip at the walls the instability appears infinitely
amplified (figure \ref{fig:evaldiverg_varslip}). A possible explanation for
this is that the instability moves into a region very close to the boundaries
as we introduce free slip and can no longer be resolved. It is clear from
figure \ref{fig:eigenmode_varslip} that the first normal stress difference
introduced via the instability moves towards the walls as $\alpha$ tends to
zero.

\begin{figure}
%    \showthe\columnwidth %use this to determine size of figures.
    \centering
    \includegraphics[width=\textwidth]{./figures/vara_ev_kx001_Wi2_log}
    \caption{
	Growth rate of an eigenvalue at $k_x=0.01$ as $\alpha$ the slip
	parameter, is reduced. The growth rate assymtotes to the no-slip
	condition.
    }
    \label{fig:evaldiverg_varslip}
\end{figure}

\begin{figure}
%    \showthe\columnwidth %use this to determine size of figures.
    \centering
    \includegraphics[width=\textwidth]{./figures/eigenmode_varslip}
    \caption{
	Eigenmodes of the first normal stress difference at $k_x=0.01$ as
	$\alpha$ the slip parameter, is reduced. As the slip at the walls is
	reduced, the first normal stress difference of the instablity becomes
	more and more localised at the walls.
}
    \label{fig:eigenmode_varslip}
\end{figure}


\section{Discussion and conclusion}\label{sec:conclusion}

We have shown that the velocity profile for the plane Couette viscoelastic
self-sustaining process is susceptible to a viscoelastic lift up effect and
does become streaky in the purely elastic regime. This streaky base profile is
linearly unstable, with a range of wave numbers between about $k_x = 0$ and
$k_x = 0.04$. The eigenmodes of this instability are localised in the centre of
the channel.


%\begin{align}
%    Re \left[ \frac{\partial \mathbf{v}}{\partial t} + \mathbf{v} \cdot \nabla  \mathbf{v} \right] &= - \nabla p + \beta \nabla^{2} \mathbf{v} + \frac{1-\beta}{Wi} \nabla \cdot \mathbf{\tau} \\
%    \nabla \cdot \mathbf{v} &= 0 \\
%    \dot{\tau}/Wi + \overset{\nabla}\tau &= \left(\nabla \mathbf{v}\right)^{T} + \nabla{\mathbf{v}}
%\end{align}

%\begin{equation}
%\beta \nabla^{2}_{2} \overline{V}^{x}  = - \mathit{Re} \overline{(\mathbf{v} \cdot \nabla) v}^{x} + \frac{\partial p}{\partial y} - \left(1-\beta \right) \left(\overline{\frac{\partial T_{yy}}{\partial y}}^{x} + \overline{\frac{\partial T_{yz}}{\partial z}}^{x} \right)
%\end{equation}

\pagebreak
\bibliographystyle{jfm}
% Note the spaces between the initials

\bibliography{SSP_paper}

\end{document}
